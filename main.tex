\documentclass{article}
\usepackage{graphicx} % Required for inserting images

\title{Implications éthiques liées à la création de logiciels

}
\author{Noheyla Younesse, Séréna Pot, Celine Aouci, Morjiane Aït-Mokhtar}
\date{April 2025}

\begin{document}

\maketitle

\section{Introduction}

\section{Dévelloppement}
\subsection{La protection de la vie privée sur les applications}
À une ère où nombre d'applications collectent nos données  comme notre nom, notre adresse, notre e-mail ou notre téléphone. Il est crucial pour les utilisateurs que notre vie privée soit un minimum protégée. Néanmoins, il subsiste un risque quant à la confidentialité des utilisateurs. Ces données peuvent alors être utilisées à des fins publicitaires, de ciblage excessif ou même, dans des cas extrêmes, de cybercriminalité (comme de la fraude ou des vols d'identité).\\
On peut donc réfléchir à des manières d'améliorer la protection de la vie privée sur ces applications. Du côté des utilisateurs et des développeurs, il y a plusieurs solutions.


\section{Conclusion}
\end{document}
