\documentclass{article}
\usepackage{graphicx} % Required for inserting images
\usepackage[utf8]{inputenc}
\usepackage{array}
\usepackage{geometry}
\geometry{margin=1in}
\usepackage{booktabs}
\usepackage{xcolor}

\title{Implications éthiques liées à la création de logiciels

}
\author{Noheyla Younesse , Séréna Pot , Celine Aouci , Morjiane Aït-Mokhtar,
main
\date{April 2025}

\begin{document}

\maketitle

\section{Introduction}
Dans notre époque actuelle marquée par le numérique, les logiciels occupent une place centrale dans notre quotidien, influençant nos interactions, nos décisions et même nos droits fondamentaux. Cependant, leur conception et leur utilisation soulèvent de nombreuses questions éthiques, notamment en matière de protection de la vie privée et d’équité algorithmique.

Les applications collectent et exploitent une quantité croissante de données personnelles, posant ainsi la question de leur sécurisation et de leur utilisation. De même, les algorithmes, bien qu’efficaces, peuvent être biaisés et reproduire des discriminations existantes, impactant ainsi divers domaines tels que la justice, l’emploi ou la publicité.

Face à ces enjeux, il est essentiel d’adopter une approche responsable dans le développement des logiciels. Cette réflexion implique non seulement les développeurs, mais aussi les législateurs et la société dans son ensemble. Ce travail explorera ainsi les principales implications éthiques liées à la création de logiciels, en mettant en lumière les défis à relever et les solutions envisageables.


\section{Développement}
\subsection{La protection de la vie privée sur les applications}
À une ère où nombre d'applications collectent nos données  comme notre nom, notre adresse, notre e-mail ou notre téléphone. Il est crucial pour les utilisateurs que notre vie privée soit un minimum protégée. Néanmoins, il subsiste un risque quant à la confidentialité des utilisateurs. Ces données peuvent alors être utilisées à des fins publicitaires, de ciblage excessif ou même, dans des cas extrêmes, de cybercriminalité (comme de la fraude ou des vols d'identité).\\
On peut donc réfléchir à des manières d'améliorer la protection de la vie privée sur ces applications. Du côté des utilisateurs et des développeurs, il y a plusieurs solutions. Intéressons-nous dans un premier temps aux utilisateurs :\\
- ils peuvent restreindre et vérifier les accès accordés aux applications (caméra, micro, localisation, etc..)\\
- privilégier des applications respectueuses de la vie privée (DuckDuckGo ou Signal par ex)\\
- désactiver le suivi publicitaire et ajuster les paramètres d'application.\\
- réduire l'usage de connexions via Google, Facebook ou Apple pour limiter la centralisation des données \\
- Utiliser des VPN et bloqueurs de trackers\\
Ensuite voyons le cas des développeurs : \\
- Ne collecter que les infos nécessaires \\
- Stocker et transmettre les données de manière sécurisée (chiffrement de bout en bout, HTTPS..)
- Informer clairement les utilisateurs sur les données collectées et leur donner un contrôle total sur leur utilisation. \\
- Respecter le RGPD (Europe), la CCPA (Californie) et d’autres lois sur la protection des données. \\
\begin{table}[h!]
\centering
\renewcommand{\arraystretch}{1.4}
\begin{tabular}{>{\bfseries}m{3.0cm} m{2.0cm} m{3.5cm} m{4.5cm} m{2.5cm}}
\toprule
Loi & Zone géographique & Objectifs principaux & Droits des utilisateurs & Sanctions possibles \\
\midrule
RGPD \newline (Règlement Général sur la Protection des Données) 
& Union européenne 
& Protéger les données personnelles et harmoniser les règles à l’échelle européenne 
& \begin{itemize}
    \item Consentement explicite
    \item Droit d'accès, rectification, suppression
    \item Portabilité des données
    \item Opposition au traitement
\end{itemize}
& Jusqu’à 20 M€ ou 4\% du CA mondial \\
\midrule
CCPA \newline (California Consumer Privacy Act) 
& Californie (États-Unis) 
& Donner aux consommateurs le contrôle sur leurs données personnelles 
& \begin{itemize}
    \item Savoir quelles données sont collectées
    \item Demander la suppression
    \item Refuser la vente des données
\end{itemize}
& Jusqu’à 7 500 \$ par violation intentionnelle \\
\midrule
Loi ePrivacy \newline (directive européenne en cours de réforme) 
& Union européenne 
& Encadrer la confidentialité des communications électroniques (cookies, emails, etc.) 
& \begin{itemize}
    \item Consentement pour les cookies
    \item Confidentialité des communications
    \item Opposition au spam
\end{itemize}
& Dépend des législations nationales \\
\bottomrule
\end{tabular}
\caption{Comparatif des principales réglementations sur la protection des données}
\end{table}

Ces lois cherchent à restaurer un équilibre entre innovation numérique et respect de la vie privée. Elles responsabilisent les entreprises développant des applications et redonnent aux utilisateurs le contrôle sur leurs données personnelles. Toutefois, leur efficacité dépend de leur mise en œuvre concrète, de la vigilance des utilisateurs et du rôle actif des autorités de régulation. Néanmoins, il est important pour les utilisateurs comme pour les développeurs de rester vigilants sur l'importance de la protection des données dans les applications et du côté des développeurs de donner un maximum de choix à l’utilisateur quant aux données qu'il souhaite fournir ou non à l'application. Néanmoins il ne faut pas que l'utilisateur,lui, oublie qu'il doit rester informé et lire toutes les informations mises à sa disposition sur ce sujet. La protection des données est donc un sujet central dans la création d'application.






\subsection{les biais algorithmiques} 
Les biais algorithmiques en informatique désignent des erreurs systématiques dans les décisions prises par des algorithmes, souvent causées par des données biaisées, des choix de conception involontaires, des erreurs d’interprétation ou des boucles de rétroaction biaisées. Ces biais reflètent ou amplifient les discriminations existantes et peuvent se manifester dans divers domaines. Ils trouvent leur origine dans plusieurs facteurs. D'abord, les données biaisées, qui sont parfois non représentatives ou porteuses de discriminations historiques, notamment dans les statistiques socio-économiques ou raciales. Ensuite, les choix de conception, où les développeurs influencent involontairement les résultats en programmant selon leurs propres préjugés inconscients ou en sélectionnant certaines variables.aussi les boucles de rétroaction biaisées renforcent ces inégalités, car les décisions prises par un algorithme influencent les données futures, perpétuant ainsi les discriminations initiales. Ces biais se manifestent dans plusieurs domaines. En justice pénale, par exemple, les outils d’évaluation des risques pénalisent davantage certains groupes, exacerbant les disparités raciales. En reconnaissance faciale, les systèmes sont souvent moins précis pour les personnes à la peau foncée, en raison d’un manque de diversité dans les bases de données d'entraînement. De même, dans la publicité ciblée, certains emplois mieux rémunérés sont davantage proposés aux hommes qu’aux femmes. Pour limiter ces biais, plusieurs solutions existent. Il est essentiel de diversifier et nettoyer les ensembles de données afin qu’ils reflètent mieux la diversité de la population. Les algorithmes doivent être régulièrement audités pour identifier et corriger d’éventuels biais. Par ailleurs, la mise en place de cadres légaux est nécessaire pour garantir la transparence et l’équité dans l’utilisation des intelligences artificielles.Il est primordial d’impliquer des experts en éthique et en diversité dès la conception des systèmes afin d’anticiper et de limiter ces effets négatifs.Donc même si les biais algorithmiques représentent un défi majeur, une approche proactive combinant amélioration technique, régulation et expertise multidisciplinaire peut contribuer à les atténuer et à rendre l’intelligence artificielle plus juste et équitable.




\section{Dévelloppement}
\subsection{La protection de la vie privée sur les applications}
À une ère où nombre d'applications collectent nos données  comme notre nom, notre adresse, notre e-mail ou notre téléphone. Il est crucial pour les utilisateurs que notre vie privée soit un minimum protégée. Néanmoins, il subsiste un risque quant à la confidentialité des utilisateurs. Ces données peuvent alors être utilisées à des fins publicitaires, de ciblage excessif ou même, dans des cas extrêmes, de cybercriminalité (comme de la fraude ou des vols d'identité).\\
On peut donc réfléchir à des manières d'améliorer la protection de la vie privée sur ces applications. Du côté des utilisateurs et des développeurs, il y a plusieurs solutions.

\section{Développement}
\subsection{Implications éthiques liées à la sécurité  logiciels}
Dans un monde de plus en plus numérisé, où les entreprises et les organismes utilisent de
nombreux logiciels, la sécurité des logiciels est devenue un enjeu majeur. En effet, chaque
logiciel conçu peut contenir des vulnérabilités susceptibles d’être exploitées à des fins
malveillantes, certains sont même créés à ce titre. Selon le rapport annuel 2024 du ministère de
l’intérieur sur la cybercriminalité, il y aurait
278 770 atteintes numériques enregistrées en
2023. Ces statistiques soulignent la fragilité de notre espace numérique et posent
d'importantes questions éthiques liées à la sécurité.\\

La conception de logiciels ne peut se limiter à une approche technique : elle engage la
responsabilité éthique des développeurs, des chefs de projet, des entreprises et même des
états qui les autorisent. Comment garantir la sécurité des utilisateurs ? Comment respecter leur
vie privée ? Quelles sont les conséquences et les inconvénients ?\\

La sécurité des logiciels fait référence à toutes les mesures et les précautions adoptées pour
créer des logiciels sécurisés. Comme le souligne Khoury (2021),
« la sécurité logicielle désigne
l’ensemble des moyens visant à concevoir des logiciels aussi sécuritaires que possible,
notamment en réduisant leur surface d’attaque et en assurant la protection des données
manipulées ».\\

Cette approche repose sur une démarche proactive consistant à anticiper les risques avant leur
surgissement. Elle implique d'intégrer la sécurité dès la phase de conception, de réaliser des
analyses de code régulières, de contrôler les accès aux fonctionnalités sensibles, et de sécuriser
les interfaces de programmation. Il s’agit de protéger les applications tout au long de leur cycle
de vie – conception, développement, déploiement et maintenance.\\

Un enjeu majeur est la protection des données personnelles, y compris des informations
sensibles comme l’identité, les photos ou les messages privés. En cas de faille, ces informations
pourraient être mises à profit par des cybercriminels. Ils ont la possibilité d'exploiter ces
données sensibles pour reconnaître une personne et usurper son identité, ce qui représente une
atteinte à la vie privée et pourrait causer de graves dommages à la réputation d'un individu.
Cela pose des problèmes de sécurité, mais également de responsabilité morale\\

Sur le plan juridique, un certain nombre de lois encadrent la cybersécurité. En France et au sein
de l’U.E, le Règlement Général sur la Protection des Données (RGPD) oblige les entreprises à
protéger les données des utilisateurs et à signaler toute violation dans un délais strict. En
parallèle, la Loi de Programmation Militaire (LPM) et la Directive NIS 2 intensifient les exigences
de sécurité pour les secteurs vitaux.\\

La responsabilité éthique en matière de sécurité logicielle incombe à divers intervenants.
Conformément au Code d'éthique ACM/IEEE (ACM, 2018), les développeurs sont tenus de
rédiger un code sûr, de garantir la confidentialité des données dès la phase de conception et
d'agir dans l'intérêt du public. Ils doivent rapporter toute vulnérabilité identifiée et ne doivent
en aucun cas intégrer des fonctionnalités malicieuses ou dissimulées. Les chefs de projet, pour
leur part, sont chargés d’incorporer la sécurité dès la planification, d’assurer une transparence
lors de la gestion des incidents et de favoriser un esprit d'équipe éthique, comprenant
l'éducation des membres du personnel sur les questions de cybersécurité. Enfin, les entreprises
ont une responsabilité globale : elles doivent se conformer aux lois (RGPD, LPM, etc.), adopter
des normes telles que l’ISO 27001, formaliser des politiques éthiques (ex. : Charte Numérique
Responsable), et garantir la transparence en cas d’incident pour maintenir la confiance des
usagers.\\

Les failles de sécurité logicielle peuvent avoir des conséquences techniques (interruption
de service, vol de données), sociales (perte de confiance, atteinte à la vie privée), et juridiques
(amendes, responsabilité civile). Selon l’ANSSI, ces vulnérabilités exposent les systèmes à des
attaques critiques, comme les ransomwares, qui peuvent paralyser une entreprise entière.\\

Au-delà de la conformité, un développeur ou un expert en cybersécurité dispose
généralement des compétences pour exploiter des failles. Toutefois, une interrogation
fondamentale se pose alors en référence à la philosophie morale
"Ce n’est pas parce que je le
peux, que je dois le faire." L’éthique exige de faire des choix responsables, même si la loi ne les
proscrit pas explicitement\\

Face à un dilemme éthique et en prévention, il est utile de se poser plusieurs questions : Est-ce
légal ? Est-ce conforme au code de conduite de mon entreprise ? Cela pourrait-il nuire à
quelqu’un, à l’entreprise ou à sa réputation ? Que penseraient les autres si cela faisait la une
des journaux ?
Si les réponses soulèvent un doute, alors il vaut mieux s’abstenir ou demander conseil à un
responsable ou à un juriste. Un exemple typique : dissimuler volontairement des informations
dans un rapport de test d’intrusion. Ce n’est peut-être pas illégal, mais ce n’est pas éthique. Et
cela peut entraîner de lourdes conséquences.\\


\section{Conclusion}
L’éthique dans la création de logiciels est un enjeu fondamental à l’ère du numérique. La protection de la vie privée et la lutte contre les biais algorithmiques illustrent bien les défis auxquels les développeurs et les utilisateurs doivent faire face. Alors que la collecte et l’exploitation des données personnelles nécessitent des garanties solides pour éviter tout abus, les biais algorithmiques rappellent l’importance d’une conception inclusive et transparente des systèmes intelligents.
Pour répondre à ces défis, des efforts conjoints entre les développeurs, les régulateurs et la société civile sont nécessaires. L’adoption de pratiques responsables, la mise en place de cadres légaux appropriés et l’implication d’experts en éthique permettent de limiter les dérives et de garantir une utilisation plus équitable des technologies. Ainsi, le développement logiciel doit toujours intégrer une réflexion éthique afin de respecter les principes fondamentaux de justice, de transparence et de respect des droits des individus.
\end{document}
