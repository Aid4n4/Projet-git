\documentclass{article}
\usepackage{graphicx} % Required for inserting images
\usepackage[utf8]{inputenc}
\usepackage{array}
\usepackage{geometry}
\geometry{margin=1in}
\usepackage{booktabs}
\usepackage{xcolor}

\title{Implications éthiques liées à la création de logiciels

}
\author{Noheyla Younesse , Séréna Pot , Celine Aouci , Morjiane Aït-Mokhtar,
main
\date{April 2025}

\begin{document}

\maketitle

\section{Introduction}
Dans notre époque actuelle marquée par le numérique, les logiciels occupent une place centrale dans notre quotidien, influençant nos interactions, nos décisions et même nos droits fondamentaux. Cependant, leur conception et leur utilisation soulèvent de nombreuses questions éthiques, notamment en matière de protection de la vie privée et d’équité algorithmique.

Les applications collectent et exploitent une quantité croissante de données personnelles, posant ainsi la question de leur sécurisation et de leur utilisation. De même, les algorithmes, bien qu’efficaces, peuvent être biaisés et reproduire des discriminations existantes, impactant ainsi divers domaines tels que la justice, l’emploi ou la publicité.

Face à ces enjeux, il est essentiel d’adopter une approche responsable dans le développement des logiciels. Cette réflexion implique non seulement les développeurs, mais aussi les législateurs et la société dans son ensemble. Ce travail explorera ainsi les principales implications éthiques liées à la création de logiciels, en mettant en lumière les défis à relever et les solutions envisageables.


\section{Développement}
\subsection{La protection de la vie privée sur les applications}
À une ère où nombre d'applications collectent nos données  comme notre nom, notre adresse, notre e-mail ou notre téléphone. Il est crucial pour les utilisateurs que notre vie privée soit un minimum protégée. Néanmoins, il subsiste un risque quant à la confidentialité des utilisateurs. Ces données peuvent alors être utilisées à des fins publicitaires, de ciblage excessif ou même, dans des cas extrêmes, de cybercriminalité (comme de la fraude ou des vols d'identité).\\
On peut donc réfléchir à des manières d'améliorer la protection de la vie privée sur ces applications. Du côté des utilisateurs et des développeurs, il y a plusieurs solutions. Intéressons-nous dans un premier temps aux utilisateurs :\\
- ils peuvent restreindre et vérifier les accès accordés aux applications (caméra, micro, localisation, etc..)\\
- privilégier des applications respectueuses de la vie privée (DuckDuckGo ou Signal par ex)\\
- désactiver le suivi publicitaire et ajuster les paramètres d'application.\\
- réduire l'usage de connexions via Google, Facebook ou Apple pour limiter la centralisation des données \\
- Utiliser des VPN et bloqueurs de trackers\\
Ensuite voyons le cas des développeurs : \\
- Ne collecter que les infos nécessaires \\
- Stocker et transmettre les données de manière sécurisée (chiffrement de bout en bout, HTTPS..)
- Informer clairement les utilisateurs sur les données collectées et leur donner un contrôle total sur leur utilisation. \\
- Respecter le RGPD (Europe), la CCPA (Californie) et d’autres lois sur la protection des données. \\
\begin{table}[h!]
\centering
\renewcommand{\arraystretch}{1.4}
\begin{tabular}{>{\bfseries}m{3.0cm} m{2.0cm} m{3.5cm} m{4.5cm} m{2.5cm}}
\toprule
Loi & Zone géographique & Objectifs principaux & Droits des utilisateurs & Sanctions possibles \\
\midrule
RGPD \newline (Règlement Général sur la Protection des Données) 
& Union européenne 
& Protéger les données personnelles et harmoniser les règles à l’échelle européenne 
& \begin{itemize}
    \item Consentement explicite
    \item Droit d'accès, rectification, suppression
    \item Portabilité des données
    \item Opposition au traitement
\end{itemize}
& Jusqu’à 20 M€ ou 4\% du CA mondial \\
\midrule
CCPA \newline (California Consumer Privacy Act) 
& Californie (États-Unis) 
& Donner aux consommateurs le contrôle sur leurs données personnelles 
& \begin{itemize}
    \item Savoir quelles données sont collectées
    \item Demander la suppression
    \item Refuser la vente des données
\end{itemize}
& Jusqu’à 7 500 \$ par violation intentionnelle \\
\midrule
Loi ePrivacy \newline (directive européenne en cours de réforme) 
& Union européenne 
& Encadrer la confidentialité des communications électroniques (cookies, emails, etc.) 
& \begin{itemize}
    \item Consentement pour les cookies
    \item Confidentialité des communications
    \item Opposition au spam
\end{itemize}
& Dépend des législations nationales \\
\bottomrule
\end{tabular}
\caption{Comparatif des principales réglementations sur la protection des données}
\end{table}

Ces lois cherchent à restaurer un équilibre entre innovation numérique et respect de la vie privée. Elles responsabilisent les entreprises développant des applications et redonnent aux utilisateurs le contrôle sur leurs données personnelles. Toutefois, leur efficacité dépend de leur mise en œuvre concrète, de la vigilance des utilisateurs et du rôle actif des autorités de régulation. Néanmoins, il est important pour les utilisateurs comme pour les développeurs de rester vigilants sur l'importance de la protection des données dans les applications et du côté des développeurs de donner un maximum de choix à l’utilisateur quant aux données qu'il souhaite fournir ou non à l'application. Néanmoins il ne faut pas que l'utilisateur,lui, oublie qu'il doit rester informé et lire toutes les informations mises à sa disposition sur ce sujet. La protection des données est donc un sujet central dans la création d'application.






\subsection{les biais algorithmiques} 
Les biais algorithmiques en informatique désignent des erreurs systématiques dans les décisions prises par des algorithmes, souvent causées par des données biaisées, des choix de conception involontaires, des erreurs d’interprétation ou des boucles de rétroaction biaisées. Ces biais reflètent ou amplifient les discriminations existantes et peuvent se manifester dans divers domaines. Ils trouvent leur origine dans plusieurs facteurs. D'abord, les données biaisées, qui sont parfois non représentatives ou porteuses de discriminations historiques, notamment dans les statistiques socio-économiques ou raciales. Ensuite, les choix de conception, où les développeurs influencent involontairement les résultats en programmant selon leurs propres préjugés inconscients ou en sélectionnant certaines variables.aussi les boucles de rétroaction biaisées renforcent ces inégalités, car les décisions prises par un algorithme influencent les données futures, perpétuant ainsi les discriminations initiales. Ces biais se manifestent dans plusieurs domaines. En justice pénale, par exemple, les outils d’évaluation des risques pénalisent davantage certains groupes, exacerbant les disparités raciales. En reconnaissance faciale, les systèmes sont souvent moins précis pour les personnes à la peau foncée, en raison d’un manque de diversité dans les bases de données d'entraînement. De même, dans la publicité ciblée, certains emplois mieux rémunérés sont davantage proposés aux hommes qu’aux femmes. Pour limiter ces biais, plusieurs solutions existent. Il est essentiel de diversifier et nettoyer les ensembles de données afin qu’ils reflètent mieux la diversité de la population. Les algorithmes doivent être régulièrement audités pour identifier et corriger d’éventuels biais. Par ailleurs, la mise en place de cadres légaux est nécessaire pour garantir la transparence et l’équité dans l’utilisation des intelligences artificielles.Il est primordial d’impliquer des experts en éthique et en diversité dès la conception des systèmes afin d’anticiper et de limiter ces effets négatifs.Donc même si les biais algorithmiques représentent un défi majeur, une approche proactive combinant amélioration technique, régulation et expertise multidisciplinaire peut contribuer à les atténuer et à rendre l’intelligence artificielle plus juste et équitable.




\section{Dévelloppement}
\subsection{La protection de la vie privée sur les applications}
À une ère où nombre d'applications collectent nos données  comme notre nom, notre adresse, notre e-mail ou notre téléphone. Il est crucial pour les utilisateurs que notre vie privée soit un minimum protégée. Néanmoins, il subsiste un risque quant à la confidentialité des utilisateurs. Ces données peuvent alors être utilisées à des fins publicitaires, de ciblage excessif ou même, dans des cas extrêmes, de cybercriminalité (comme de la fraude ou des vols d'identité).\\
On peut donc réfléchir à des manières d'améliorer la protection de la vie privée sur ces applications. Du côté des utilisateurs et des développeurs, il y a plusieurs solutions.

\section{Conclusion}
L’éthique dans la création de logiciels est un enjeu fondamental à l’ère du numérique. La protection de la vie privée et la lutte contre les biais algorithmiques illustrent bien les défis auxquels les développeurs et les utilisateurs doivent faire face. Alors que la collecte et l’exploitation des données personnelles nécessitent des garanties solides pour éviter tout abus, les biais algorithmiques rappellent l’importance d’une conception inclusive et transparente des systèmes intelligents.
Pour répondre à ces défis, des efforts conjoints entre les développeurs, les régulateurs et la société civile sont nécessaires. L’adoption de pratiques responsables, la mise en place de cadres légaux appropriés et l’implication d’experts en éthique permettent de limiter les dérives et de garantir une utilisation plus équitable des technologies. Ainsi, le développement logiciel doit toujours intégrer une réflexion éthique afin de respecter les principes fondamentaux de justice, de transparence et de respect des droits des individus.
\end{document}
